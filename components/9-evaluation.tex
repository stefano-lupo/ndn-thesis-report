\chapter{Evaluation}\label{sec:eval}
In order to validate the design and evaluate the performance of \game{}, several tests were performed in a variety of scenarios. As there a large number of parameters which impact the overall performance, each test was designed to isolate a single parameter to determine the impact the parameter has on the game's performance.

As the performance of \game{} is heavily dependent on the NDN topology, all tests were repeated using several topologies where appropriate.

The performance is also impacted by the game mechanics which are enabled and the automation script used. As described in \refsec{sec:impl:automation}, the automation script uses a fixed random seed to allow for a direct comparison between tests using the same topology.

As updates to the status of players makes up the significant majority of the traffic seen on the network, the evaluation of \game{} provided here will only consider the traffic generated from the \textit{PlayerStatus} updates. The results gathered based on the other data types (\textit{blocks} and \textit{projectiles}) were extremely similar to those gathered using the \textit{PlayerStatus}. Thus, as there is substantially more data available when considering the traffic generated by the \textit{PlayerStatus} updates, this data type was favoured. 

However, in all tests, \textbf{all game mechanics were enabled} and players were able to move, place blocks and shoot projectiles. Thus, all of the data used by \game{} was generated during these tests, but the \textit{block} and \textit{projectile} data is omitted from the evaluation for brevity.  

All tests were repeated several times and each test instance produced data that was extremely similar. In the interest of clarity, error bars are omitted from the figures as they offer no additional insight.

The evaluation is split into two stages - \refsec{sec:eval:metrics-testing} examines the impacts of tweaking parameters on the metrics discussed in \refsec{sec:impl:metrics} and \refsec{sec:eval:scalability} examines how \game{} scales in a real-world scenario.


\section{Metric Testing}\label{sec:eval:metrics-testing}

\subsection{Parameters}
The backend implementation of \game{} requires several parameters to be defined, many of which can have a substantial impact on performance. Each parameter used in the evaluation is outlined below.

\subsubsection*{Interest Timeout (s)}
The length of time in seconds that a consumer will wait for an Interest to be satisfied before retransmitting the same Interest.

\subsubsection*{Data Freshness Period (ms)}
The amount of time a Data packet is considered \textit{fresh} after being received by an NFD instance. This is defined on a hop-by-hop basis.

\subsubsection*{Dead Reckoning (DR) Pub Throttling}
This mechanism reduces the amount of updates a publisher will produce by maintaining a view of the game object as seen by the consumers. This is outlined further in \refsec{sec:des:dr}. There are two parameters associated with this mechanism - whether or not it is enabled, and the error threshold which results in an update being produced. The error threshold is a measure of distance, and uses \textit{game world units (GWU)}, and a description of a GWU is given below.


\subsubsection*{Interest Management (IM)}
This mechanism reduces the rate at which a consumer will express Interests for a given piece of data based on how far away the game object is from the consumer. This mechanism requires four parameters - whether or not it is enabled, the full interest radius ($r_{full}$) in \textit{GWU}, the minimum interest radius ($r_{min}$) in \textit{GWU} and the maximum sleep time $MAX\_SLEEP$ in seconds. These parameters were discussed previously in \refsec{sec:des:im}. 

\subsubsection*{Publisher Update Rate \textit{(PUR)} (Hz)}
This defines the rate at which publishers will check to see if their game object(s) should be updated. This does \textbf{not} directly trigger updates to be sent and instead informs the backend module of whether or not there is an update available. If DR publisher throttling is disabled, the publishers will inform the backend of an update at the defined rate. If it is enabled, the publishers \textit{check} to see if an update is required, at the defined rate. 

\subsubsection*{Publisher Queue Check Rate \textit{(PQR)} (Hz)}
As outlined in \refsec{sec:des:sync-protocol}, when Interests arrive, they are added to the \textit{outstanding interests} data structure. This parameter defines the rate at which publishers will attempt to satisfy these Interests. At the defined rate, the publisher will examine each of the Interests in this data structure, check to see if an update has been published by the game engine and satisfy the Interest if possible. 

\subsection{Parameter-Metric Matrix}
The metrics used for the evaluation are the \textit{round-trip-time (RTT)}, \textit{Interest aggregation factor (IAF)}, \textit{cache rate} and \textit{position delta}, each of which are discussed in \refsec{sec:impl:metrics}.

The \textit{parameter-metric matrix} is shown in \reftbl{tbl:eval:metric-matrix}. This table shows the theoretical impact on each of the metrics when a parameter is \textbf{increased}. The \textit{+}, \textit{0} and \textit{-} values in the cells represent an increase, no change and a decrease to the associated metric. Finally, the resulting impact on the overall performance of the game is indicated by the colour of the cell, where green, white and red indicate a positive, neutral and negative impact on the performance respectively. 

\begin{table}[H]
    \centering
    \begin{tabular}{@{}lcccc@{}}
        & \multicolumn{4}{c}{\textbf{Impact on Metric}}                                                                                                               \\
    \multirow{-2}{*}{\textbf{Test Parameters}} & \multicolumn{1}{l}{\textbf{RTT}} & \multicolumn{1}{l}{\textbf{IAF}} & \multicolumn{1}{l}{\textbf{Cache Rate}} & \multicolumn{1}{l}{\textbf{Position Delta}} \\
    Interest Timeout (ms)                      & 0                                & 0                                & 0                                       & \cellcolor[HTML]{FFCCC9}+                   \\
    Data Freshness Period (ms)                 & \cellcolor[HTML]{9AFF99}-        & 0                                & \cellcolor[HTML]{9AFF99}+               & \cellcolor[HTML]{FFCCC9}+                   \\
    DR Pub Throttling Enabled                  & \cellcolor[HTML]{FFCCC9}+        & \cellcolor[HTML]{9AFF99}+        & \cellcolor[HTML]{9AFF99}+               & \cellcolor[HTML]{FFCCC9}+                   \\
    DR Pub Throttling Threshold (GWU)          & \cellcolor[HTML]{9AFF99}-        & \cellcolor[HTML]{9AFF99}+        & 0                                       & \cellcolor[HTML]{FFCCC9}+                   \\
    IM Enabled                                 & 0                                & \cellcolor[HTML]{FFCCC9}-        & \cellcolor[HTML]{9AFF99}+               & \cellcolor[HTML]{FFCCC9}+                   \\
    IM Full Radius, $r_{full}$ (GWU)              & 0                                & \cellcolor[HTML]{9AFF99}+        & \cellcolor[HTML]{FFCCC9}-               & \cellcolor[HTML]{9AFF99}-                   \\
    IM Min Radius, $r_{min}$ (GWU)                & 0                                & \cellcolor[HTML]{9AFF99}+        & \cellcolor[HTML]{FFCCC9}-               & \cellcolor[HTML]{9AFF99}-                   \\
    IM Max Sleep Time, $MAX\_SLEEP$ (s)          & 0                                & \cellcolor[HTML]{FFCCC9}-        & \cellcolor[HTML]{9AFF99}+               & \cellcolor[HTML]{FFCCC9}+                   \\
    Pub Update Rate (Hz)                           & \cellcolor[HTML]{FFCCC9}-        & \cellcolor[HTML]{FFCCC9}-        & \cellcolor[HTML]{FFCCC9}-               & \cellcolor[HTML]{9AFF99}-                   \\
    Pub Queue Check Rate (Hz)                      & \cellcolor[HTML]{FFCCC9}-        & \cellcolor[HTML]{FFCCC9}-        & \cellcolor[HTML]{FFCCC9}-               & \cellcolor[HTML]{9AFF99}-                  
    \end{tabular}
    \centering
    \caption{Configurable parameters and their associated impacts on the metrics. \textit{+}, \textit{0} and \textit{-} represent an increase, no change and a decrease to the metric respectively when the parameter is \textbf{increased}. The colour of the cells represents whether this has a net positive (green), neutral (white) or negative (red) impact on the performance of \game{}.}
    \label{tbl:eval:metric-matrix}
\end{table}

For all tests the following parameter values were unchanged:
\begin{gather*}
    Interest\ Timeout = 1000ms\\
    Pub\ Update\ Rate = 30Hz\\
    Pub\ Queue\ Check\ Rate = 60Hz
\end{gather*}

For convenience of viewing, the testing topologies discussed in \refsec{sec:impl:topologies} are repeated in \reffig{fig:eval:topologies}.

\begin{figure}[H]
    \centering
    \figsize{assets/impl/topologies.png}{0.9}
    \caption{Topologies used in evaluation of \game{}}
    \label{fig:eval:topologies}
\end{figure}

An important note to make here is in the magnitude of a \textit{GWU}. GWUs are used to by LibGDX to remove the need to consider different resolution sizes. To provide some context, in \game{}, the width of a player avatar is 1 GWU as shown in \reffig{fig:eval:gwus}

\begin{figure}[H]
    \centering
    \figsize{assets/eval/GWUs.png}{0.7}
    \caption{The width of a player avatar in \game{} is 1 GWU}
    \label{fig:eval:gwus}
\end{figure}

Finally, each of the tests were allowed to run for approximately 5 minutes, as the results obtained did not appear to change once the test reached the steady state.

\subsection{No Caching, no IM and no DR Publisher Throttling}
In order to provide a benchmark for future tests, the simplest form of \game{} was tested in which no IM or DR publisher throttling was used and caching was disabled.

\subsubsection*{Round-Trip-Times (RTT)}
The first metric examined was RTT. Each of the four topologies performed similarly in this regard. For this reason, the RTTs are only shown for the \textit{dumbbell} topology, though the others can be found in \refapp{app:eval:basic}. As seen in \reffig{fig:eval:basic:rtt}, the RTT for the \textit{PlayerStatus} of each of the nodes appears to be approximately normally distributed around the 20 - 40 ms range, with a moderate positive skew. This follows from the \textit{publisher update rate (PUR)} of 30Hz, meaning updates are published every 33ms as no optimizations were used. Thus, the RTT is heavily dependent on the \textit{PUR}, as expected. The RTT can be reduced by increasing the \textit{PUR}, however this will increase the traffic on the network and thus reduce the scalability of \game{}.  

\begin{figure}[H]
    \centering
    \figsize{assets/eval/basic/rtt/dumbbell.png}{0.8}
    \caption{\textit{RTTs} for \textit{PlayerStatus} in the \textit{dumbbell} topology}
    \label{fig:eval:basic:rtt}
\end{figure}


\subsubsection*{Position Deltas}
As caching, IM and DR publisher throttling are disabled, consumers are receiving a constant stream of the most up to date player positions. This suggests that the \textit{position deltas} should be at their lowest in these tests. However, as the optimizations are disabled, these tests will have the largest amount of traffic on the network. Thus, provided the network is not overwhelmed, the \textit{position deltas} should be extremely low in this case. Otherwise, the lack of network optimizations may affect performance sufficiently that the \textit{position deltas} will be large. 

As with the RTTs, each of the topologies performed very similarly in this regard. Again, only the histogram for the \textit{dumbbell} topology is shown and all others are given in \refapp{app:eval:basic}.

\begin{figure}[H]
    \centering
    \figsize{assets/eval/basic/pos-deltas/dumbbell.png}{0.9}
    \caption{\textit{Position deltas} for players in the \textit{dumbbell} topology. Note the frequencies over \textit{position deltas} over 1.5 GWUs are extremely low, but are non-zero and were left in the histograms to highlight this fact.}
    \label{fig:eval:basic:pos-deltas}
\end{figure}

The \textit{position deltas} seen in \reffig{fig:eval:basic:pos-deltas} show that player positions are synchronized very closely. The figure shows that almost 90\% of the time, when a position update was received for a player, the extrapolated position was within half a \textit{GWU} of the new position. The figure also shows that the \textit{position delta} can spike as high as 3 \textit{GWUs}, although this is extremely rare. As these tests were performed on a single machine, this may be a result of CPU contention amongst the Docker containers.



\subsubsection*{Interest Aggregation Factor (IAF)}
As previously discussed, one of the main benefits of NDN in a MOG scenario is Interest aggregation. The extent of the Interest aggregation in \game{} is shown in \reffig{fig:eval:basic:interest-agg-linear} and \reffig{fig:eval:basic:interest-agg-dumbbell}. In these figures, the orange bar represents the total number of Interests expressed for data belonging to a particular node, and the blue bar represents the total number of Interest that node received. 

\begin{figure}[H]
    \centering
    \figsize{assets/eval/basic/interest-agg/linear.png}{0.75}
    \caption{Interest aggregation for \textit{PlayerStatus} in the \textit{linear} topology. The \textit{IAF} of both nodes is 1 as expected.}
    \label{fig:eval:basic:interest-agg-linear}
\end{figure}

As expected, when there are only two nodes, no Interest aggregation occurs. Thus, the orange and blue bars match in \reffig{fig:eval:basic:interest-agg-linear} and both nodes had an \textit{Interest Aggregation Factor (IAF)} of 1.

\begin{figure}[H]
    \centering
    \figsize{assets/eval/basic/interest-agg/dumbbell.png}{0.75}
    \caption{Interest aggregation for \textit{PlayerStatus} in the \textit{dumbbell} topology. The \textit{IAF} of nodes A, B, C and D were calculated to be 0.45, 0.46, 0.45 and 0.44 respectively}
    \label{fig:eval:basic:interest-agg-dumbbell}
\end{figure}


The \textit{dumbbell} topology on the other hand shows that a substantial amount of Interest aggregation takes place, with all nodes having an \textit{IAF} below 0.5. This means that all of the nodes had to service less than half of the requests that would otherwise be required in IP based solution. Again, the other topologies performed very similarly and the graphs are provided in \refapp{app:eval:basic}.

\subsection{Effects of Enabling Caching}
As outlined in \refsec{sec:sota:mog-ndn-benefits}, caching can provide benefits to \game{} in the case where a node falls behind the Interest aggregation period, which is quite possible with a publisher update frequency of 30Hz. However, one of the challenges with caching in NDN is that the \textit{freshness period} (cache lifetime) is defined on a hop-by-hop basis. Thus, \textit{freshness periods} must be kept very low. To examine the impact of enabling caching, the tests were repeated, with the only difference being that the Data packets were given a \textit{freshness period} of 20ms. IM and DR publisher throttling both remained disabled for these tests and the \textit{linear} topology was omitted as caching will not have any impact in the case of only 2 players.

By enabling caching, the rate of Interests seen by producers in the \textit{tree} topology decreased by 2 Interests per second for all nodes, as seen in \reffig{fig:eval:caching:interest-impact-tree}. The results for the \textit{square} and \textit{dumbbell} were very similar and can be found in \refapp{app:eval:caching}.

\begin{figure}[H]
    \centering
    \figsize{assets/eval/caching/interest-rate-impact/tree.png}{0.75}
    \caption{Impact of caching on the Interest rates seen by producers in the \textit{tree} topology}
    \label{fig:eval:caching:interest-impact-tree}
\end{figure}

These reductions are a direct result of enabling caching. Nodes which fall behind the Interest aggregation period previously resulted in the Interest propagating all the way to the producer. However, by enabling caching, the Interests from nodes which fall behind are now served by intermediate routers, resulting in a decrease in network traffic.

By enabling caching, the IAF becomes less clear as it is not possible to determine whether the Interests never reached the producer due to caching or Interest aggregation. In either case it is a major benefit to the performance of the game. However, as the same test had previously been performed without caching, a reasonable comparison can be made. The reduction factor in the case of the \textit{dumbbell} topology with caching enabled was 0.33 for all nodes, as seen in \reffig{fig:eval:caching:interest-agg-dumbbell}.  

\begin{figure}[H]
    \centering
    \figsize{assets/eval/caching/interest-aggregations/dumbbell.png}{0.75}
    \caption{Difference between the number of Interests expressed by consumers and seen by producers in the \textit{dumbbell} topology when caching was enabled with a \textit{freshness period} of 33ms. The IAF of all nodes was 0.33.}
    \label{fig:eval:caching:interest-agg-dumbbell}
\end{figure}

The values shown in \reffig{fig:eval:caching:interest-agg-dumbbell} can be compared to \reffig{fig:eval:basic:interest-agg-dumbbell}, which shows the results when caching was disabled. In the case of no caching, the \textit{IAFs} were 0.45, 0.46, 0.45 and 0.44 for nodes A, B, C and D respectively. Thus, enabling caching further improved the reduction factor by between 0.11 and 0.13.

The cache rate of a particular node is heavily dependent on the node's position in the topology. For example, caching essentially only occurs at the intermediate router nodes in \textit{tree} topology. Consider an Interest whose corresponding Data packet has already been produced. This Interest must flow through several Intermediate routers who will all have a cached copy that is \textbf{fresher} than the copy at the producer's cache. This is due to the fact that the Data will be cached in the producer's CS first, and will thus become non-fresh there first. Thus, all intermediate routers will typically be able to serve a cached copy of a Data packet for longer than the producer will. 

However, there is an exception to this when the intermediate router's CS becomes full and the Data must be evicted. This will happen more frequently in intermediate routers than in any of the leaf (player) nodes of the \textit{tree} topology as the router nodes must cache Data requested by multiple nodes, whereas the leaf nodes only cache the Data they request. Thus, it is possible than an intermediate router can forward an Interest to a leaf node and have it satisfied by the leaf nodes cache.

\begin{figure}[H]
    \centering
    \figsize{assets/eval/caching/cache-rates/tree.png}{0.75}
    \caption{Cache rates by node for all nodes in the \textit{tree} topology. Note that the cache rates for nodes A, B, C and D are non-zero.}
    \label{fig:eval:caching:cache-rate-tree}
\end{figure}

The cache rates shown in \reffig{fig:eval:caching:cache-rate-tree} agrees with the expected behaviour previously described. Although these cache rates appear rather low, this is due to the fact that most Interests are aggregated and multicasted, instead of being satisfied by a Data packet in the CS.

Interestingly, node G, the root node, also essentially has a non-zero cache rate. Consider the case where A expresses an Interest for a piece of data owned by C (see \reffig{fig:eval:topologies}). This Interest will be forwarded through E, G and F until it reaches C. Once C produces the Data, it is then cached at each hop on the way back. At this point, only B or E will express an Interest for the same piece of data, as A has the data and C is the producer. If B then requests the same packet, the Interest will be satisfied by E, and if D requests the same packet, the Interest will be satisfied by F. Thus, G very rarely serves any cached Data, though it is possible in the case of CS eviction as described previously. 

The above results indicate that caching is a major benefit to \game{} from a network performance point of view, however the impact it has on the gameplay must also be considered. In topologies with a large number of hops, it is possible that even relatively small values for \textit{freshness period} can result in players being served extremely old Data packets (see \refsec{sec:conc:fw}).


\subsection{Effects of Enabling DR Publisher Throttling (DRPT)}
In order to test DR publisher throttling, the tests were again repeated. The 20ms \textit{freshness period} was used once again meaning caching was enabled. The DR publisher throttling threshold was set to 0.5 GWU, which is half the width of the player's avatar. 

As with the previous tests, the \textit{dumbbell}, \textit{square} and \textit{tree} topologies performed very similarly. For this reason, only the results for the \textit{dumbbell} topology are shown here and the others can be found in \refapp{app:eval:pub-throt}.

Recall that at a rate of 30Hz (the publisher update rate), the game will check to see if an update needs to be produced. As publisher throttling was enabled for these tests, an update will only be produced if the player's velocity changed (\textit{velocity}), there was no entry in the recent player status cache (\textit{null}), or the DR threshold is exceeded (\textit{threshold}), as discussed in \refsec{sec:des:dr}. Otherwise, the update is skipped (\textit{skip}). The distribution of the results from the publisher update checks is shown in \reffig{fig:eval:caching:dr-pub-throt:counters:dumbbell} for the \textit{dumbbell} topology, and the equivalent figures for the other topologies are given in \refapp{app:eval:pub-throt}.

\begin{figure}[H]
    \centering
    \figsize{assets/eval/dr-pub-throt/counters/dumbbell.png}{0.75}
    \caption{Distribution of results from publisher update checks in the \textit{dumbbell} topology. Note there were no \textit{null} updates required in the test for this topology}
    \label{fig:eval:caching:dr-pub-throt:counters:dumbbell}
\end{figure}

As seen in \reffig{fig:eval:caching:dr-pub-throt:counters:dumbbell}, \textbf{almost half} of the updates can be skipped by enabling DR publisher throttling. This has a substantial impact on the network traffic as player position updates make up the majority of the traffic.

However, as with enabling caching, the impact on the game experience must be considered, and DRPT has the potential to hugely impact the game experience in a negative way, as only half of the player updates are now being published. To examine the impact on the game experience, the \textit{position deltas} were re-examined. Again, the results proved to be very similar across all topologies as they all use the same automation script and thus the results for all but the \textit{dumbbell} topology are omitted.

\begin{figure}[H]
    \centering
    \figsize{assets/eval/dr-pub-throt/position-deltas/dumbbell.png}{0.75}
    \caption{\textit{Position deltas} with DR publisher throttling enabled for \textit{dumbbell} topology.}
    \label{fig:eval:caching:dr-pub-throt:pos-deltas:dumbbell}
\end{figure}

The \textit{position deltas} shown in \reffig{fig:eval:caching:dr-pub-throt:pos-deltas:dumbbell} are very similar to those shown in \reffig{fig:eval:basic:pos-deltas}, which shows \textit{position deltas} for the benchmark test were all optimizations were disabled. This indicates that there is very little impact on the game experience when DR publisher throttling is enabled with a threshold of 0.5 GWU.

As approximately half of the updates were skipped due to DRPT, the rate of Interests seen by the publishers also dropped accordingly, as seen in \reffig{fig:eval:dr-pub-throt:dumbbell-interest-rates} and \reffig{fig:eval:dr-pub-throt:linear-interest-rates} for the \textit{dumbbell} and \textit{linear} topologies respectively. Unlike with enabling caching, enabling DR publisher throttling does affect the Interest rates in the \textit{linear} topology. The corresponding figures for the \textit{square} and \textit{tree} topologies are provided in \refapp{app:eval:pub-throt}.

\begin{figure}[H]
    \centering
    \figsize{assets/eval/dr-pub-throt/interest-rate-impact/dumbbell.png}{0.75}
    \caption{Effects of enabling caching with a \textit{freshness period} of 20ms, and DR publisher throttling (\textit{DRPT}) with a tolerance of 0.5 GWU on the Interest rate seen by nodes in the \textit{dumbbell} topology.}
    \label{fig:eval:dr-pub-throt:dumbbell-interest-rates}
\end{figure}
 

\begin{figure}[H]
    \centering
    \figsize{assets/eval/dr-pub-throt/interest-rate-impact/linear.png}{0.75}
    \caption{Effects of enabling caching with a \textit{freshness period} of 20ms, and DR publisher throttling (\textit{DRPT}) with a tolerance of 0.5 GWU on the Interest rate seen by nodes in the \textit{linear} topology.}
    \label{fig:eval:dr-pub-throt:linear-interest-rates}
\end{figure}



\section{Scalability Testing}\label{sec:eval:scalability}
In order to test the scalability of \game{}, a new topology was built which contained 16 players and four intermediate routers. The topology is shown in \reffig{fig:eval:scalability-topology}.

\begin{figure}[H]
    \centering
    \figsize{assets/eval/scalability/scalability-topology.png}{0.6}
    \caption{The topology used for testing the scalability of \game{}, consisting of 4 intermediate routers (\textit{Q-T}) arranged in a \textit{square} topology, with 4 player nodes behind each router (\textit{A-P}).}
    \label{fig:eval:scalability-topology}
\end{figure}

As the topology consists of 20 nodes, initial testing showed that the performance was constrained by the computing power available on the machine used for development. As such, the topology was deployed onto a cluster of 10 Amazon Web Service instances, using a Docker swarm as described in \refsec{sec:impl:test:docker}.

The game world was also altered to support the larger number of players. The width and height of the game world were doubled to provide space for the increased number of players, and extra impassable boundaries were also added in the form of squares in each of the four corners of the game world. This was done to encourage hot spotting, which is when the density of players is very high in certain areas of the game world. This is a common issue faced by MOGs that can cause scalability problems. The adjusted game world is shown in \ref{fig:eval:hotspotting}.

\begin{figure}[H]
    \centering
    \figsize{assets/eval/scalability/hotspotting.png}{0.75}
    \caption{A screenshot of the adjusted game world during scalability testing. The screenshot shows the extra boundaries added to the game world, the occurrence of hot spots in the game world and the critical interest regions defined by $r_{full}$ and $r_{min}$ used by the Interest management system.}
    \label{fig:eval:hotspotting}
\end{figure}


\subsection{Benchmark with DRPT and IM Disabled}
As before, the test was conducted with some of the optimizations disabled to provide a benchmark. In this case, \textit{DR publisher throttling (DRPT)} and \textit{Interest management (IM)} were disabled, and caching was enabled with a \textit{freshness period} of 20ms as before. The \textit{RTTs} and \textit{Position Deltas} for each of the 16 players are shown in \reffig{fig:eval:no-im-no-dr:agg-packet-times} and \reffig{fig:eval:no-im-no-dr:agg-pos-deltas} respectively and the effect of Interest aggregation is shown in \reffig{fig:eval:no-im-no-dr:int-agg}.

\begin{figure}[H]
    \centering
    \figsize{assets/eval/scalability/no-im-no-dr/aggregated-packet-times.png}{0.75}
    \caption{\textit{RTTs} for each of the nodes in the scalability test with no IM or DRPT. As before, there is a peak around 33ms which corresponds to the publisher update frequency of 30Hz.}
    \label{fig:eval:no-im-no-dr:agg-packet-times}
\end{figure}

\begin{figure}[H]
    \centering
    \figsize{assets/eval/scalability/no-im-no-dr/aggregated-position-deltas.png}{0.75}
    \caption{\textit{Position Deltas} for each of the nodes in the scalability test with no IM or DRPT. As before, the values are very close to zero in most cases indicating the players' positions are very tightly synchronized, even with 16 players.}
    \label{fig:eval:no-im-no-dr:agg-pos-deltas}
\end{figure}

As seen in the figures above, \game{} scales up to 16 players very well and the results are very similar to those shown in \reffig{fig:eval:basic:rtt} and \reffig{fig:eval:basic:pos-deltas}, which show the benchmark results for the \textit{dumbbell} topology which only contained four players. 

Finally, the IAF of each node in the scalability is shown in \reffig{fig:eval:no-im-no-dr:int-agg}. The IAFs ranged from 0.07-0.09 in the scalability test, which shows that the Interest aggregation mechanisms scales appropriately with the number of players. 

\begin{figure}[H]
    \centering
    \figsize{assets/eval/scalability/no-im-no-dr/interest-aggregations.png}{0.75}
    \caption{The effect of Interest aggregation in the scalability test with no IM and no DRPT. The \textit{Interest aggregation factor (IAF)} ranged between 0.07 and 0.09 in this case, which is significantly lower than the best \textit{IAF} seen in previous tests were optimizations were enabled. However, \textit{IAFs} are expected to decrease as more players are added, due to the overall larger number of Interests expressed.}
    \label{fig:eval:no-im-no-dr:int-agg}
\end{figure}



\subsection{Enabling Dead Reckoning Publisher Throttling (DRPT)}
With a benchmark established, \textit{DRPT} was then re-enabled with the same values used in previous tests. 

As \textit{DRPT} effectively slows down the publisher update rate, the expected effect of enabling DRPT is an increasing in RTTs, and this is seen in \reffig{fig:eval:dr:agg-packet-times}. It is worth noting that this is \textbf{not} indicative of poorer network performance, as the increase in RTT is fully explained by the reduction in effective publisher update rates.

\begin{figure}[H]
    \centering
    \figsize{assets/eval/scalability/dr/aggregated-packet-times.png}{0.75}
    \caption{\textit{RTTs} for each of the nodes in the scalability test with DRPT enabled and IM disabled. The peak increases from the previous value of 33ms due to the effective decrease in the rate of publisher updates, caused by DRPT.}
    \label{fig:eval:dr:agg-packet-times}
\end{figure}



The other expected outcome of enabling DRPT is a reduction in the rate of Interests seen by producers. This is due to the fact that the effective publisher update rate decreases, meaning the effective rate of Interests seen by producers also decreases. 

As seen in previous tests, enabling \textit{DRPT} with a tolerance of 0.5 GWU allowed producers to skip between 40-50\% of the updates, as seen in \reffig{fig:eval:dr:dr-pub-throt}. This is very similar to the results of tests with fewer players, indicating that the DRPT mechanism also scales appropriately with the number of players.

\begin{figure}[H]
    \centering
    \figsize{assets/eval/scalability/dr/dr-pub-throt.png}{0.75}
    \caption{The distribution of the results of publisher update checks for each of the players in the \textit{scalability test}. Approximately 40-50\% of the updates could be skipped as a result of \textit{DRPT}.}
    \label{fig:eval:dr:dr-pub-throt}
\end{figure}

Finally, the effective reduction in the rate of Interests seen by producers as a result of skipping 40-50\% of the updates can be seen in \reffig{fig:eval:dr:interest-rate-impacts}.

\begin{figure}[H]
    \centering
    \figsize{assets/eval/scalability/dr/interest-rate-impacts.png}{0.75}
    \caption{The reduction in Interest rates seen by producers in the scalability test as a result of enabling \textit{DRPT}. The figure shows a reduction of approximately 40-50\% which aligns with the skipped updates seen in \reffig{fig:eval:dr:dr-pub-throt}.}
    \label{fig:eval:dr:interest-rate-impacts}
\end{figure}


\subsection{Enabling Interest Management (IM)}
\textit{IM} enables players to intelligently control the rate at which they express Interests for remote object updates, based on the distance the object is from the player. \textit{DRPT} was once again disabled to examine the impact of IM alone. IM was enabled using the following parameters, which were defined in \refsec{sec:des:im}:

\begin{gather*}
    r_{full} = 10\ GWU\\
    r_{min} = 20\ GWU\\
    MAX\_SLEEP = 2\ seconds
\end{gather*}

There are two expected outcomes of enabling IM. The first is a significant reduction in the rate of Interests seen by publishers, due to the reduced rate of Interests expressed by consumers. The second is a likely increase in \textit{position deltas} due to the longer time between updates, during which remote players may have changed direction. The results of the test agree with these two expected outcomes and can be seen in \reffig{fig:eval:no-im-no-dr:agg-pos-deltas} and \reffig{fig:eval:im:interest-impacts} respectively.

\begin{figure}[H]
    \centering
    \figsize{assets/eval/scalability/im/aggregated-position-deltas.png}{0.75}
    \caption{The substantial increase in \textit{position deltas} as a result of enabling IM. This is due to consumers reducing the rate at which they express Interests for updates to remote objects which are far away. }
    \label{fig:eval:im:pos-deltas}
\end{figure}


\begin{figure}[H]
    \centering
    \figsize{assets/eval/scalability/im/interest-rate-impacts.png}{0.75}
    \caption{The reduction in Interest rates seen by producers in the scalability test as a result of enabling \textit{IM}. Note that DRPT was not enabled for this test.}
    \label{fig:eval:im:interest-impacts}
\end{figure}

\subsection{Enabling DRPT and IM}
Finally, the test was repeated a fourth time to examine the impacts of enabling both DRPT and IM. Although IM will diminish the effectiveness of DRPT to a certain degree, the two optimizations are entirely compatible. The combined use of DRPT and IM should be the most beneficial in terms of reducing Interest rates, and the results shown in \reffig{fig:eval:im-dr:impacts} agree with this theory. 

\begin{figure}[H]
    \centering
    \figsize{assets/eval/scalability/im-dr/interest-rate-impacts.png}{1}
    \caption{The impacts of enabling IM, DR and both IM and DR on the Interest rates seen by producers in the scalability tests.}
    \label{fig:eval:im-dr:impacts}
\end{figure}


There are a small number of exceptions in which one of the optimizations out performs the combination. An example of this is the difference in Interest rates seen by node C in \reffig{fig:eval:im-dr:impacts} as a result of enabling and disabling the optimizations. IM alone caused a greater reduction in Interest rate for node C than the combination of IM and DRPT. These discrepancies are most likely a result of small differences in how the game played out. For example, if node C ended up becoming very isolated from other players, IM would cause it to see a much-reduced Interest rate. 

