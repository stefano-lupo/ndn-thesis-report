\begin{appendices}

\chapter{Invertible Bloom Filters}\label{app:IBF}


IBFs are an extension to standard BFs which replace the simple bit array used in BFs with a list of objects. IBFs extend BFs to support element retrieval and deletion. The indices produced by the hash functions are used as indices into this list in order to extract the objects of interest for a given element. The objects in the IBF list contain a \textit{key}, a \textit{value} and a \textit{count}. IBF operations are defined as follows, where an \textit{o} refers to an object at index $i$ such that $i$ is the output of hashing the element with one of the hash functions:

\begin{labeling}{insert(key, val)  }
    \item[\textit{insert(key, val)}] For each \textit{o}, \textit{o.key := o.key XOR key}, the new value becomes \textit{o.value:= o.value XOR val}, \textit{o.count++}. 
    \item[\textit{delete(key)}] Assuming the element had been inserted, for each \textit{o}, the new key becomes $existingKey\;XOR\;deleteKey$, the new value becomes $existingValue\;XOR\;deleteValue$ and the count is decremented.
    \item[\textit{get(key)}] There are three cases to consider when retrieving an \textit{value} by \textit{key}:
    \begin{itemize}
        \item If the \textit{count} of \textbf{any} of the objects of interest are zero, the element was never inserted. 
        \item If none of the objects of interest have a \textit{count == 1}, the element cannot be retrieved but may have been inserted. \item If any of the objects of interest have a \textit{key} which matches the \textit{key} to be retrieved, then the \textit{value} of that object is returned. Otherwise, the element was never inserted.
    \end{itemize}
\end{labeling}





\chapter{Docker Compose file for players A and B}\label{app:docker-two-players}
The most basic topology is two players who are directly connected to one and other and this is shown in \reffig{fig:app:docker-l2}. 

\begin{figure}[H]
    \centering
    \figsize{assets/app/linear-top.png}{0.8}
    \caption{Players A and B directly connected via a single link}
    \label{fig:app:docker-l2}
\end{figure}

A simplified version of the resulting Docker Compose file generated for testing this topolgoy is shown in \refcode{lst:app:docker-l2}. Of particular interest are the \textit{enviroment} and \textit{network} fields for each of the two services. 

The \textit{environment} field specifies the location of the NLSR config file that the node should use and the command the node should use to start \game{} with the appropriate parameters.

The \textit{network} field specifies the name of the overlay network to connect to and the alias that this service should be given on that overlay network. This allows the node to be accessible at this alias by other nodes in the overlay network.


\lstinputlisting[
  caption={Docker Compose file for a two player topology},
  label={lst:app:docker-l2},
  style=javaStyle
]{code/docker/linear-two-compose.yml}




\end{appendices}