\chapter*{Abstract}
\subsection*{Multiplayer Online Game (MOG) Communication using Named Data Networking (NDN)}
\subsubsection*{Stefano Lupo, MAI Computer Engineering}
\subsubsection*{Supervised by Dr Stefan Weber}
\subsubsection*{April 2019}
NDN is a new Internet architecture in which the core abstraction changes from sending data between two specific hosts in a network, to naming self-contained pieces of data, and allowing that data to be fetched by the name alone. NDN aims to replace IP as the \textit{universal network layer} of tomorrow's Internet, and supports running over a variety of lower level protocols such as Ethernet, Bluetooth and even IP itself. With the ever-increasing popularity of MOGs, this research aimed to investigate the possibility of using NDN as the core communication protocol for a multiplayer game.

As the data generated in MOGs is so varied, one of the main goals of this research was to determine how well NDN can support all of these different data types, some of which fit nicely into a content-oriented abstraction, and some of which are entirely host oriented. The other major challenge associated with MOGs is that they require extremely high-performance networking solutions that provide high bandwidth and low latency.

The research led to the development of a novel protocol for synchronizing remote game objects across all players in the game using NDN. This protocol was designed to exploit the three main benefits of NDN - \textit{interest aggregation}, \textit{native multicast} and \textit{in-network caching}. A peer-to-peer 2D game was implemented and used to test the performance of the synchronization protocol. The results show that NDN is an excellent choice for MOG communications and that the developed synchronization protocol could easily support 16 concurrent players, while providing the performance required for a fast paced MOG. 